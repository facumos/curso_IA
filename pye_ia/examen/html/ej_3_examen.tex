
% This LaTeX was auto-generated from MATLAB code.
% To make changes, update the MATLAB code and republish this document.

\documentclass{article}
\usepackage{graphicx}
\usepackage{color}

\sloppy
\definecolor{lightgray}{gray}{0.5}
\setlength{\parindent}{0pt}

\begin{document}

    
    
\subsection*{Contents}

\begin{itemize}
\setlength{\itemsep}{-1ex}
   \item c) Intervalo de confianza
   \item c) Test de Hip�tesis
\end{itemize}


\subsection*{c) Intervalo de confianza}

\begin{verbatim}
randn('seed', 1234)
% numero de simulaciones
n = 2e4;
% numero de muestras
N = 10;
% media muestral
% la usamos como media de distribucion en el ejercicio
mu = 48;
% dispersion
sigma = 4;
% el intervalo de confianza de 95
% teorico es:
mu_min_teorico = mu - 1.96*sigma/sqrt(N);
mu_max_teorico = mu + 1.96*sigma/sqrt(N);
% nivel de confianza del intervalo, deberia ser igual a 0.95 al final de
% la simulacion
confianza = 0;
for i=1:n
    % generar distribucion
    % esta surge de la estimacion de media muestral
    X = sigma/sqrt(N)*randn(N,1) + mu;
    %  chequeo cuantos valores estan en el intervalo de confianza
    confianza = confianza + (1/(n*N)) *sum(X>=mu_min_teorico & X<=mu_max_teorico);
end
% estimacion del intervalo de confianza
confianza_simulado = confianza
confianza_teorico = 0.95
\end{verbatim}

        \color{lightgray} \begin{verbatim}
confianza_simulado =

    0.9504


confianza_teorico =

    0.9500

\end{verbatim} \color{black}
    

\subsection*{c) Test de Hip�tesis}

\begin{par}
CIMT - Hypothesis Testing, ejemplo de las bebidas \begin{verbatim}http://www.cimt.org.uk/projects/mepres/alevel/stats_ch10.pdf\end{verbatim} la hipotesis nula es que los participantes eligen al azar
\end{par} \vspace{1em}
\begin{verbatim}
p = 1/3;
% nivel de significacion
alpha = 0.05;
% numero de personas
N = 10;
% minimo numero de personas
N_min = 0;
% probabilidad de necesitar i personas para el ensayo
p_i_personas = 0;
% cuantas personas son necesarias para rechazar la hipotesis nula,
%busco desde el maximo hacia el minimo de personas
for i=N:-1:0
    % buscar el numero de personas hasta con la cdf binomial hasta que se
    %exceda alpha, y retornar el numero de personas anterior
    % probabilidad de que 'i' personas deban acertar el ensayo
    p_i_personas = 0;
    % si i personas aciertan, i+1 incluyen a las i personas previas
    % asi que las debo incluir
    for j=i:N
        p_i_personas = p_i_personas + binopdf(j,N,p);
    end;
        % si esta probabilidad excede a alpha, entonces necesitamos una
        % persona mas en el ensayo
        if(p_i_personas>alpha)
            % devolver el numero de personas previo
            % (recordar que i es descendiente)
            N_min = i+1;
            break;
        end;
end;
            disp('Minimo numero de personas:'), disp(N_min)
\end{verbatim}

        \color{lightgray} \begin{verbatim}Minimo numero de personas:
     7

\end{verbatim} \color{black}
    


\end{document}
    
